\documentclass[12pt]{article}
\usepackage[latin1]{inputenc}
\usepackage{amsmath,amsthm, amsfonts, amscd , amssymb}

\usepackage[spanish]{babel}
\usepackage{hyperref}
\usepackage[dvipsnames, usenames]{color}
\usepackage{framed}

\usepackage{latexsym}

\newtheorem{theorem}{Teorema}[section]
\newtheorem{lemma}{Lema}[section]
% \theoremstyle{definition}
\newtheorem{notation}{Notaci?n}[section]
\newtheorem{example}{Ejemplo}[section]
\newtheorem{proposition}{Proposici?n}[section]
\newtheorem{ejemplo}[theorem]{Ejemplo}
\newtheorem{corollary}{Corolario}[section]
\newtheorem{remark}{Observaci\'on}[section]
\newtheorem{ejercicio}{Ejercicio}[section]
\newtheorem{algorithm}{Algoritmo}[section]
\newtheorem{corolario}{Corolario}[section]

\newtheorem{definition}{Definici\'on}
\setlength{\textwidth}{6.5in} \setlength{\oddsidemargin}{0in}
\topmargin 0pt \textheight 7.75in
\newcommand{\convprob}{ \buildrel{p}\over\longrightarrow}


\def \EE {\mathbb{E}}
\def \PP {\mathbb{P}}

\begin{document}

\centerline{\bf ESTAD\'ISTICA (Q)}%

\medskip


{\sc Simulaci\'on - Teorema Central del L?mite  - Clase 10}

\bigskip 


Antes de empezar, los invitamos a visitar el sitio \href{https://www.nature.com/collections/qghhqm/pointsofsignificance}{Point of Significance}, una publicaci\'on de Nature dedicada a la divulgaci?n de la estad?stica dentro de las ciencias naturales. 
En particular, los invitamos a que miren el trabajo  \href{https://www.nature.com/articles/nmeth.2613
} {Importance of being uncertain}, considerando que vamos a querer replicar parte de los resultados presentados en la Figura 3. 


A lo largo de esta Gu?a estudiaremos emp?ricamente la distribuci\'{o}n del promedio de
variables aleatorias independientes e id\'{e}nticamente distribuidas. A trav\'{e}s de
los histogramas correspondientes, analizaremos el comportamiento de la distribuci?n del promedio a medida que aumentamos $n$, la cantidad de variables a promediar. 

Para ello generaremos un conjunto de $n$  datos  con una
distribuci\'{o}n dada y luego calcularemos su promedio. 
Replicaremos \'{e}sto mil veces, es decir, generaremos $Nrep=1000$  realizaciones de la variable aleatoria  $\overline{X}_n$,  para diferentes valores de $n$.  Observemos que, en principio,
desconocemos la distribuci\'{o}n de $\overline{X}_n$. Utilizando  las $Nrep=1000$ realizaciones del promedio 
realizaremos un histograma de los promedios generados para
obtener una aproximaci\'{o}n de la densidad o la funci\'{o}n de probabilidad
de $\overline{X}_n$.

\section{Teorema Central del L?mite: El Teorema}
Antes de empezar con nuestras simulaciones, recordemos 
el  Teorema  Central del L?mite. Sea $(W_i)_{i\geq 1}$ una sucesi\'on de v.a.i.i.d.
con $\mathbb E(W_i)=\mu$ y $\mathbb{V}(X_i)=\sigma^2$. 

Una de las maneras de ver el TCL en t?rminos del promedio es la siguiente: 
\begin{equation}
\overline W_n\;{\color{red}\buildrel{a}\over\approx}\;
\mathcal N(\mu,\sqrt{\sigma^2/n})\;.
\end{equation}




Si estandarizamos al promedio, obtenemos esta otra posible presentaci?n:  
\begin{equation}
\label{TCL-promedios}
\frac{\overline W_n-\mu}{\sqrt{\sigma^2/n}}\;{\color{red}\buildrel{a}\over\approx}\;
\mathcal N(0,1)\;.
\end{equation}
%donde $(W_i)_{i\geq 1}$ son variables aleatorias i.i.d. 

\section{Teorema Central del L?mite: Simulaciones}


\subsection{Distribuci?n Bernoulli}
Sean $X_i\sim \mathcal Be(p)$ siendo $p=0,2$. Utilizaremos $X$ para denotar de manera gen?rica una variable con esta misma distribuci?n.  


\begin{enumerate}	
	\item \textquestiondown Cu?nto vale $\mathbb E(X)$?
	\textquestiondown Cu?nto vale $\mathbb{V}(X)$?
	\item Guardar en el vector ber$\_$N$\_$infty= 10000 datos correspondientes a 10.000 realizaciones $\mathcal Be(p)$ con $p=0,2$ y calcular la frecuencia relativa de cada posible valor. 
	Explorar el comando \texttt{table()}. 
	
	
	\vspace{0.2cm}	
	\textbf{La distribuci?n emp?rica del promedio}
	
	\item Guardar en el  vector \texttt{promedios$\_$bernoullies}, $Nrep=1000$ promedios utilizando $n=5$ datos con distribuci?n $\mathcal Be(p)$ siendo $p=0,2$. Recordar  el comando \texttt{rbinom()}, teniendo  precauci?n con los par?metros que utiliza esta funci?n (ver \texttt{help(rbinom)}).
	\item Realizar un histograma con los promedios guardados en \texttt{promedios$\_$bernoullies}.
	\item Repetir los ?tems 2 y 3 con $n=30$ y con $n=100$. Comparar los tres histogramas obtenidos, notar la diferencia en las escalas.
	%{\color{red} Preguntar para que vean  como los histogramas quedan cada vez mas concentrados alrededor de $\mu$. PRestar atencion a la escala de los diferentes histogramas.  }
	
	\vspace{0.2cm}
	\textbf{La distribuci?n emp?rica  del Promedio \textit{estandarizado}}
	
	\item  Realizar ahora histogramas de \texttt{promedios$\_$bernoullies$\_$est} donde a los valores guardados en \texttt{promedios$\_$bernoullies} se les resta $\mu$ y se los divide por $\sqrt{\sigma^2/n}$. Utilizar $n \in \{5,30,100\}$, \textquestiondown Qu? se observa?
	
	
	\item Repetir todos los ?tems anteriores con $p=0.5$. 
	
\end{enumerate}



\subsection{Distribuci?n Uniforme}

Sean $X_i\sim \mathcal U(a, b)$ siendo $a=67$, $b=73$. Utilizaremos $X$ para denotar de manera gen?rica una variable con esta misma distribuci?n.  

\vspace{0.3cm}
\begin{enumerate}	
	\item \textquestiondown Cu?nto vale $\mathbb E(X)$?
	\textquestiondown Cu?nto vale $\mathbb{V}(X)$?
	\item Guardar en el vector unif$\_$N$\_$infty= 10000 datos correspondientes a 10.000 realizaciones $\mathcal U(a, b)$ siendo $a=67$, $b=73$ y  realizar un histograma. 
	
	
	\vspace{0.2cm}	
	\textbf{La distribuci?n emp?rica del promedio}
	
	\item Guardar en el  vector \texttt{promedios$\_$uniformes}, $Nrep=1000$ promedios utilizando $n=5$ datos con distribuci?n $\mathcal U(a, b)$ siendo $a=67$, $b=73$. Recordar  el comando \texttt{runif()}.
	
	
	\item Realizar un histograma con los promedios guardados en \texttt{promedios$\_$uniformes}.
	\item Repetir los ?tems 2 y 3 con $n=30$ y con $n=100$. Comparar los tres histogramas obtenidos, notar la diferencia en las escalas.
	%{\color{red} Preguntar para que vean  como los histogramas quedan cada vez mas concentrados alrededor de $\mu$. PRestar atencion a la escala de los diferentes histogramas.  }
	
	\vspace{0.2cm}
	\textbf{La distribuci?n emp?rica  del Promedio \textit{estandarizado}}
	
	
		\item  Realizar ahora histogramas de \texttt{promedios$\_$uniformes$\_$est} donde a los valores guardados en \texttt{promedios$\_$uniformes} se les resta $\mu$ y se los divide por $\sqrt{\sigma^2/n}$. Utilizar  $n \in \{5,30,100\}$, \textquestiondown Qu? se observa?
	
	
\end{enumerate}





\subsection{Distribuci?n Exponencial}
Sean $X_i\sim \mathcal E(\lambda)$ siendo $\lambda=1/10$. Utilizaremos $X$ para denotar de manera gen?rica una variable con esta misma distribuci?n.  


\begin{enumerate}	
	\item \textquestiondown Cu?nto vale $\mathbb E(X)$?
	\textquestiondown Cu?nto vale $\mathbb{V}(X)$?
	
		\item Guardar en el vector exp$\_$N$\_$infty= 10000 datos con distribuci?n $\mathcal E(\lambda)$ siendo $\lambda=0,1$ y realizar un histograma de ellos. 

	
	\vspace{0.2cm}	
	\textbf{La distribuci?n emp?rica del promedio}
	
	\item Guardar en el  vector \texttt{promedios$\_$exponenciales}, $Nrep=1000$ promedios utilizando $n=5$ datos con distribuci?n $\mathcal E(\lambda)$ siendo $\lambda=0,1$. Recordar el comando \texttt{rexp()}.
	\item Realizar un histograma con los promedios guardados en \texttt{promedios$\_$exponenciales}.
	\item Repetir los ?tems 2 y 3 con $n=30$ y con $n=100$. Comparar los tres histogramas obtenidos, notar la diferencia en las escalas.
	%{\color{red} Preguntar para que vean  como los histogramas quedan cada vez mas concentrados alrededor de $\mu$. PRestar atencion a la escala de los diferentes histogramas.  }
	
	\vspace{0.2cm}
	\textbf{La distribuci?n emp?rica  del Promedio \textit{estandarizado}}
	
	
	\item  Realizar ahora histogramas de \texttt{promedios$\_$exponenciales$\_$est} donde a los valores guardados en \texttt{promedios$\_$exponenciales} se les resta $\mu$ y se los divide por $\sqrt{\sigma^2/n}$. Utilizar  $n \in \{5,30,100\}$, \textquestiondown Qu? se observa?
	
	
	
	\item  Realizar ahora histogramas de \texttt{promedios$\_$exponenciales} con $n \in \{5,30,100\}$ donde a los valores guardados en \texttt{promedios$\_$exponenciales} se les resta $\mu$ y se los divide por $\sqrt{\sigma^2/n}$. \textquestiondown Qu? se observa?
	
\end{enumerate}


\section{Simulaciones de promedios de variables aleatorias Normales}

Sean $X_i\sim \mathcal N(\mu, \sigma^2)$ siendo $\mu=70$ y $\sigma^2=1.2$. Utilizaremos $X$ para denotar de manera gen?rica una variable con esta misma distribuci?n.  



\begin{enumerate}	
	\item \textquestiondown Cu?nto vale $\mathbb E(X)$?
	\textquestiondown Cu?nto vale $\mathbb{V}(X)$?
	
	\item Guardar en el vector norm$\_$N$\_$infty= 10000 datos con distribuci?n $\mathcal N(\mu, \sigma^2)$ siendo $\mu=70$ y $\sigma^2=1.2$ y realizar un histograma de ellos. 
	
	
	\vspace{0.2cm}	
	\textbf{La distribuci?n emp?rica del promedio}
	
	\item Guardar en el  vector \texttt{promedios$\_$normales}, $Nrep=1000$ promedios utilizando $n=5$ datos con distribuci?n $\mathcal N(\mu, \sigma^2)$ siendo $\mu=70$ y $\sigma^2=1.2$. Recordar el comando \texttt{rnorm()}.
	\item Realizar un histograma con los promedios guardados en \texttt{promedios$\_$normales}.
	\item Repetir los ?tems 2 y 3 con $n=30$ y con $n=100$. Comparar los tres histogramas obtenidos, notar la diferencia en las escalas.
	%{\color{red} Preguntar para que vean  como los histogramas quedan cada vez mas concentrados alrededor de $\mu$. PRestar atencion a la escala de los diferentes histogramas.  }
	
	\vspace{0.2cm}
	\textbf{La distribuci?n emp?rica  del Promedio \textit{estandarizado}}
	
	
	\item  Realizar ahora histogramas de \texttt{promedios$\_$normales$\_$est} donde a los valores guardados en \texttt{promedios$\_$normales} se les resta $\mu$ y se los divide por $\sqrt{\sigma^2/n}$. Utilizar  $n \in \{5,30,100\}$, \textquestiondown Qu? se observa?
	
	
	
\end{enumerate}




\end{document}


\begin{enumerate}
	\item Comencemos por tomar un primer conjunto de datos de variables aleatorias
	$X_{1},\ldots,X_{1000}$ independientes con distribuci\'{o}n $U(0,1)$. Le
	pedimos a R que nos genere una muestra de ellas y luego hacemos un
	histograma. \textquestiondown A qu\'{e} densidad se parece el histograma obtenido?
	
	\item Considere dos variables aleatorias $X_{1}$ y $X_{2}$ independientes con
	distribuci\'{o}n $U(0,1)$ y el promedio de ambas, es decir,
	\[
	\overline{X}=\frac{X_{1}+X_{2}}{2}\,.
	\]
	Generando una muestra de dos variables aleatorias con distribuci\'{o}n
	$U(0,1)$, compute la variable promedio. Replique 1000 veces y a partir de los
	valores replicados realice un histograma. \textquestiondown Qu\'{e}
	caracter\'{\i}sticas tiene este histograma?
	
	\item Aumentemos a cinco las variables promediadas. Considere ahora 5
	variables aleatorias uniformes independientes,es decir $X_{1},X_{2}%
	,\dots,X_{5}$ i.i.d. con $X_{i}\sim U(0,1)$ y defina
	\[
	\overline{X}=\frac{1}{5}\sum_{i=1}^{5}X_{i}\,.
	\]
	
	
	Generando muestras de cinco variables aleatorias con distribuci\'{o}n $U(0,1)$
	compute la variable promedio. Repita 1000 veces y realice un histograma
	para los valores obtenidos. Compare con el histograma anterior.
	\textquestiondown Qu\'{e} observa?
	
	\item Aumentemos a\'{u}n m\'{a}s la cantidad de variables promediadas.
	Generando muestras de 30 variables aleatorias con distribuci\'{o}n $U(0,1)$
	repita el \'{\i}tem anterior. \textquestiondown Qu\'{e} observa?
	
	\item \'{I}dem anterior generando muestras de 500 variables aleatorias.
	\textquestiondown Qu\'{e} pasa si se aumenta el tama\~{n}o de la muestra?
	Observe que para poder comparar los histogramas de los distintos conjuntos de
	datos ser\'{a} necesario tenerlos dibujados en la misma escala tanto para el
	eje horizontal como para el vertical. 
	%Por eso, en general es m\'{a}s c\'{o}modo hacer boxplots para comparar distintos conjuntos de datos.
	
	\item Finalmente h\'agalo tambi\'{e}n para 1200.
	%y haga un boxplot de los 6 conjuntos de datos en el mismo gr\'{a}fico. En este gr\'{a}fico se ver\'{a} que a medida que aumenta el $n$ los valores de los promedios tienden a concentrarse, \textquestiondown alrededor de qu\'{e} valor? 
	Calcule media y varianza muestral para cada conjunto de datos. \textquestiondown Puede dar los
	valores te\'{o}ricos a los que deber\'{\i}an parecerse? 
	%Realice un qqplot para cada uno de los 6 conjuntos de datos. \textquestiondown Son esperables los resultados?
	
	\item El teorema central del l\'{\i}mite nos dice que cuando hacemos la
	siguiente transformaci\'{o}n con los promedios, $$\frac{\overline{X}%
		_{n}-E\left(  X_{1}\right)  }{\sqrt{\frac{Var\left(  X_{1}\right)  }{n}}}\,,$$ la
	distribuci\'{o}n de estas variables aleatorias se aproxima a la de una normal
	est\'{a}ndar, cuando $n$ es suficientemente grande. Para comprobarlo
	emp\'{\i}ricamente, hagamos esta transformaci\'{o}n en los 6 conjuntos de
	datos (es razonable hacerlo para valores de $n$ suficientemente grandes, lo
	realizaremos en todos los casos para comparar) y luego comparemos los datos
	transformados mediante histogramas.
	%y boxplots.
	
	\item{\color{red} \textquestiondown \'Esto va?} Repetir los \'{\i}tems anteriores generando ahora variables con
	distribuci\'{o}n $\mathcal{C}(0,1).$ Comparar los resultados obtenidos.
	Recordar que la densidad de una Cauchy es%
	\[
	f_{X}\left(  x\right)  =\frac{1}{\pi\left(  1+x^{2}\right)  },
	\]
	que es una densidad sim\'{e}trica alrededor del cero, con colas que acumulan
	m\'{a}s probabilidad que la normal est\'{a}ndar, y que no tiene esperanza ni
	varianza finitas.
\end{enumerate}


\end{document}




\section{Ley de los Grandes N?meros: El Teorema}
Antes de empezar con nuestras simulaciones, recordemos 
que dice la lay de los grande n?mero establece que 
\begin{equation}
\label{LGN-promedios}
\overline W_n=\frac{1}{n}\sum_{i=1}^nW_i \longrightarrow \mathbb E(W)\quad\hbox{en probabilidad}\;,
\end{equation}
donde $(W_i)_{i\geq 1}$ son variables aleatorias i.i.d. 
En particular, si las variables tienen distribuci?n Bernoulli, obtenemos que la frecuencia relativa de ?xitos converge a su probabilidad. M?s generalmente, si $(X_i)_{i\geq 1}$ son variables i.i.d. y $A$ es un conjunto de n?mero reales, podemos definir
$$Y_i=I_{X_i\in A}=I_A(X_i)$$
Es decir, $Y_i$ vale 1 si $X_i$ pertenece al conjunto $A$ y cero caso contrario. En tal caso, las variables $Y_i$ tienen distribuci?n Bernoulli, con $\mathbb P(Y_i=1)=\mathbb P(X_i\in A)$, 
y por consiguiente, $\mathbb E(Y)=\mathbb P(X\in A)$. 
Invocando la LEy de los Grandes N?meros,  utilizando $Y_i$ en lugar de $W_i$, tenemos que 
$$
\overline Y_n=\frac{1}{n}\sum_{i=1}^nY_i \longrightarrow \mathbb E(Y)\quad\hbox{en probabilidad}
$$
Es decir, 
\begin{equation}
\label{LGN-frec-rel}
\frac{1}{n}\sum_{i=1}^nI_A(X_i) \longrightarrow \mathbb P(X\in A)\quad\hbox{en probabilidad}
\end{equation}

Es decir, la frecuencia relativa converge a la probabilidad. 


\section{Ley de los Grandes N?meros: Simulaciones}


Ahora vamos a estudiar emp?ricamente  la ley de los grandes n?meros. 

Para ello, vamos  considerar $X_i\sim F$, generar $Nrep$ datos con la distribuci?n $F$.  Comparar promedios con esperanzas, como postula la ecuacion  (\ref{LGN-promedios}) y  frecuencias relativas com probabilidades, como se postula en las ecuaciones  (\ref{LGN-fre-rel}).



\subsection{Distribuci?n Uniforme}

Sean $X_i\sim \mathcal U(a, b)$ siendo $a=67$, $b=73$. Utilizaremos $X$ para denotar de manera gen?rica una variable con esta misma distribuci?n.  


\textbf{Promedios y Esperanzas}

\begin{enumerate}
	
	\item \textquestiondown Cu?nto vale $\mathbb E(X)$?
	
	\item Guardar en el  vector \texttt{muchas$\_$ uniformes} $Nrep=1000$ datos con distribuci?n $\mathcal U(a, b)$ siendo $a=-1$, $b=1$, utilizando el comando 
	\texttt{runif()}.
	
	
	
	\item Guardar en el  vector \texttt{muchos$\_$promedios$\_$ uniformes} el promedio de las primeras $n$ overvaciones, con $n=1, \ldots, Nrep$.
	
	\item Graficar $n$ (en el eje x) versus el promedio de los primeros $n$ datos. Utilizar $ylim=c(65,75)$.
	
	\item Repetir los items 2-4  $Ngen=10$ veces, agregando al gr?fico anterior anterior los promedios de cada individuo  en un color diferente por cada uno de ellos. 
	
	\item Indicar, seg?n la LGN,  cu?l es el valor l?mite de estas sucesiones. Se condice con lo que muestra el gr?fico? 
	
	\textbf{Frecuencias Relativas y Probabilidades}
	
	\item \textquestiondown Cu?nto vale $\mathbb P(68\leq X\leq 72)$?
	
	\item Guardar en el  vector \texttt{muchas$\_$ uniformes} $Nrep=1000$ datos con distribuci?n $\mathcal U(a, b)$ siendo $a=67$, $b=73$, utilizando el comando 
	\texttt{runif()}.
	
	
	
	\item Guardar en el  vector \texttt{muchos$\_$frec$\_$rel$\_$ uniformes} la frecuencia relativa con la cual el dato aparece en el intervalo pedido $(68,72)$ a lo largo de las  primeras $n$ overvaciones, con $n=1, \ldots, Nrep$.
	
	\item Graficar $n$ (en el eje x) versus la frecuencia relativa  de los primeros $n$ datos. Utilizar $ylim=c(0,1)$.
	
	\item Repetir los items 2-4  $Ngen=10$ veces, agregando al gr?fico anterior anterior las frecuencias relativas  de cada individuo  en un color diferente por cada uno de ellos. 
	
	\item Indicar, seg?n la LGN,  cu?l es el valor l?mite de estas sucesiones. Se condice con lo que muestra el gr?fico? 
	
	
	
\end{enumerate}

\newpage



\subsection{Distribuci?n Normal }

Sean $X_i\sim \mathcal N(\mu, \sigma^2)$ siendo $\mu=70$, $\sigma^2=1.2$. Utilizaremos $X$ para denotar de manera gen?rica una variable con esta misma distribuci?n.  


\textbf{Promedios y Esperanzas}

\begin{enumerate}
	
	\item \textquestiondown Cu?nto vale $\mathbb E(X)$?
	
	\item Guardar en el  vector \texttt{muchas  ormales} $Nrep=1000$ datos con distribuci?n $\mathcal N(\mu, \sigma^2)$ siendo $\mu=70$, $\sigma^2=1.2$, utilizando el comando 
	\texttt{rnorm()}.
	
	
	
	\item Guardar en el  vector \texttt{muchos$\_$promedios$\_$ normales} el promedio de las primeras $n$ overvaciones, con $n=1, \ldots, Nrep$.
	
	\item Graficar $n$ (en el eje x) versus el promedio de los primeros $n$ datos. Utilizar $ylim=c(60,80)$.
	
	\item Repetir los items 2-4  $Ngen=10$ veces, agregando al gr?fico anterior anterior los promedios de cada individuo  en un color diferente por cada uno de ellos. 
	
	\item Indicar, seg?n la LGN,  cu?l es el valor l?mite de estas sucesiones. Se condice con lo que muestra el gr?fico? 
	
	\textbf{Frecuencias Relativas y Probabilidades}
	
	\item \textquestiondown Cu?nto vale $\mathbb P(68\leq X\leq 72)$?
	
	\item Guardar en el  vector \texttt{muchas  ormales} $Nrep=1000$ datos  con distribuci?n $\mathcal N(\mu, \sigma^2)$ siendo $\mu=70$, $\sigma^2=1.2$, utilizando el comando 
	\texttt{rnorm()}.
	
	
	
	
	\item Guardar en el  vector \texttt{muchos$\_$frec$\_$rel  ormales} la frecuencia relativa con la cual el dato aparece en el intervalo pedido $(68,72)$ a lo largo de las  primeras $n$ overvaciones, con $n=1, \ldots, Nrep$.
	
	\item Graficar $n$ (en el eje x) versus la frecuencia relativa  de los primeros $n$ datos. Utilizar $ylim=c(0,1)$.
	
	\item Repetir los items 2-4  $Ngen=10$ veces, agregando al gr?fico anterior anterior las frecuencias relativas  de cada individuo  en un color diferente por cada uno de ellos. 
	
	\item Indicar, seg?n la LGN,  cu?l es el valor l?mite de estas sucesiones. Se condice con lo que muestra el gr?fico? 
	
	
	
\end{enumerate}





\subsection{Distribuci?n Normal }

Sean $X_i\sim \mathcal E(\lambda)$ siendo $\lambda=0.4$. Utilizaremos $X$ para denotar de manera gen?rica una variable con esta misma distribuci?n.  


\textbf{Promedios y Esperanzas}

\begin{enumerate}
	
	\item \textquestiondown Cu?nto vale $\mathbb E(X)$?
	
	\item Guardar en el  vector \texttt{muchas$\_$exponenciales} $Nrep=1000$ datos con distribuci?n $ \mathcal E(\lambda)$ siendo $\lambda=0.4$., utilizando el comando 
	\texttt{rexp()}.
	
	
	
	\item Guardar en el  vector \texttt{muchos$\_$promedios$\_$ exponenciales} el promedio de las primeras $n$ overvaciones, con $n=1, \ldots, Nrep$.
	
	\item Graficar $n$ (en el eje x) versus el promedio de los primeros $n$ datos. Utilizar $ylim=c(0,10)$.
	
	\item Repetir los items 2-4  $Ngen=10$ veces, agregando al gr?fico anterior anterior los promedios de cada individuo  en un color diferente por cada uno de ellos. 
	
	\item Indicar, seg?n la LGN,  cu?l es el valor l?mite de estas sucesiones. Se condice con lo que muestra el gr?fico? 
	
	\textbf{Frecuencias Relativas y Probabilidades}
	
	\item \textquestiondown Cu?nto vale $\mathbb P( X> 7)$?
	
	\item Guardar en el  vector \texttt{muchas$\_$exponenciales} $Nrep=1000$ datos con distribuci?n $ \mathcal E(\lambda)$ siendo $\lambda=0.4$., utilizando el comando 
	\texttt{rexp()}.
	
	
	
	
	\item Guardar en el  vector \texttt{muchos$\_$frec$\_$rel$\_$exponenciales} la frecuencia relativa con la cual el dato aparece en el intervalo pedido $(7, \infty)$ a lo largo de las  primeras $n$ overvaciones, con $n=1, \ldots, Nrep$.
	
	\item Graficar $n$ (en el eje x) versus la frecuencia relativa  de los primeros $n$ datos. Utilizar $ylim=c(0,1)$.
	
	\item Repetir los items 2-4  $Ngen=10$ veces, agregando al gr?fico anterior anterior las frecuencias relativas  de cada individuo  en un color diferente por cada uno de ellos. 
	
	\item Indicar, seg?n la LGN,  cu?l es el valor l?mite de estas sucesiones. Se condice con lo que muestra el gr?fico? 
	
	
	
\end{enumerate}


\newpage

\item\label{conjunta} En una cierta poblaci\'{o}n, se elige
un trabajador mayor de 30 a\~{n}os. Sean

$X=$ cantidad de a\~{n}os de educaci\'{o}n que recibi\'{o},

$Y=$ salario que cobra (en miles de pesos).

Se sabe que la funci\'{o}n de probabilidad puntual del vector aleatorio
$\left(  X,Y\right)  $ est\'{a} dado por $p_{XY}\left(  x,y\right)  $%
\[%
\begin{tabular}
[c]{c|cccc}%
$Y/X$ & 7 & 12 & 18 & 24\\\hline
4 & 0.14 & 0.23 & 0.02 & 0.01\\
10 & 0.06 & 0.16 & 0.25 & 0.03\\
15 & 0 & 0.01 & 0.03 & 0.06
\end{tabular}
\
\]
(es decir que, por ejemplo, $0.23=p_{XY}\left(  12,4\right)  ).$

\begin{enumerate}
	\item Hallar las funciones de probabilidad puntual $p_{X}$ y $p_{Y}$.
	
	\item Calcular $\mathbb{E}\left(X\right)$ y $\mathbb{E}\left(Y\right).$
	
	
	\item \textquestiondown Son las variables $X$ e $Y$ independientes?
	
	
		%(Interpretaci\'{o}n de covarianza) 
	
	\end{enumerate}


\item \label{bacterias} El n\'umero de colonias de bacterias de Tipo I que aparecen despu\'es de dos d\'ias de cultivo 
es una variable aleatoria Poisson  de par\'ametro $\lambda_1$, mientras que para las bacterias de Tipo II 
el par\'ametro es $\lambda_2$. 
En un laboratorio  se cultivan  simult?neamente bacterias  Tipo I y Tipo II, que  crecen de manera independiente.
Calcular la probabilidad de que al cabo de dos d\'ias aparezca 
una colonia. Utilice los par\'ametros $\lambda_1$ y $\lambda_2$ que figuran en la planilla {\color{red} ,,,,,}


\item {\color{red}Personalizar los datos para este ejercicio: 	La vida de una bater?a  se distribuye normalmente
con media de $\mu$  d?as y desviaci?n est?ndar 
$\sigma$ d?as. 
Se ponen $n$  de tales bater\'ias  en cierto equipo, de forma tal que en cuanto una bater\'ia deja de funcionar, se activa inst?ntaneamente la siguiente.  
\textquestiondown  Cu\'al es la probabilidad de que el equipo funcione al menos $t$ d\'ias? La idea es preguntar por que dura $t=2*\sqrt{n \sigma^2 }+n\mu$  }



\item {\color{red} Personalizar el $\sigma^2$.} Sean $X_1, \ldots, X_n$ i.i.d., $X_i\sim \mathcal  N(\mu, \sigma^2)$.
Encontrar $n_0$ de forma tal que $\forall n\geq n_0$
$$
P\left(\vert\overline X_n-\mu\vert>0.1\right)\leq 0.2
$$


\section{Bonus Track - Para seguir pensando}
\item Considerar nuevamente el enunciado del ejercicio 1.
\begin{enumerate}
	\item Calcular $E(XY)$.
	\item Para esta poblaci\'{o}n, \textquestiondown las variables $X$ e $Y$
	est\'{a}n positivamente asociadas?
	\item Calcular la probabilidad de que un individuo  gane \$12000 o m\'as.  
	\item Calcular la probabilidad de que un individuo con
	10 a\~nos de educaci\'on gane \$12000 o m\'as.  
\end{enumerate}
\item Considerar nuevamente el enunciado del ejercicio 2.  Sabiendo que despu\'es de dos d\'ias de cultivo se obtuvo una colonia, 
calcular la probabilidad de que sea de Tipo I utilizando los par\'ametros de $\lambda_1$ y $\lambda_2$ que le 
hemos asignado. 

\item Considerar nuevamente el enunciado del ejercicio 4:  sean $X_1, \ldots, X_n$ i.i.d., $X_i\sim \mathcal  N(\mu, \sigma^2)$, con $\sigma^2$ personalizado en la planilla {\color{red} poner donde esta}.
Encontrar $n_0$ de forma tal que $\forall n\geq n_0$
$$
P\left(\vert\overline X_n-\mu\vert>0.01\right)\leq 0.2
$$


\end{enumerate}

\end{document}





